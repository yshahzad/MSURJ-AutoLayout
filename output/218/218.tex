
        \documentclass{msurj}
        \usepackage{multirow}
        \begin{document}
        \twocolumn[
        \begin{@twocolumnfalse}
        \maketitle
        {Elodie Nickner}
        {Asymmetry of Pain-Induced Facial Grimacing\strut\\}
        {01/09/2024}
        {Research Article}
        {Department of Psychology, McGill University}
        {mice, pain, grimacing, lateralization, asymmetry }
        {xyz@mail.mcgill.ca}
        {\label{abstract}

Pain has two main components, sensory-discriminative: the quality, intensity and location of pain, and the motivational-affective: the emotional aversiveness of pain reflective of suffering\textsuperscript{1,2}. A plethora of translational preclinical and clinical measures for the sensory-discriminative component exist (e.g., von Frey, cold pressor). However, few existing measures capture the more elusive motivational-affective component, and those that do are hampered as they are not translatable across species. Post-lesion evaluation of facial grimacing of emotion-related areas of the brain suggests that the Mouse Grimace Scale is reflective of the motivational-affective component\textsuperscript{3}. Facial expressions of emotion (e.g., fear, anger) are lateralized such that the left side of the face exhibits facial expressions more strongly than the right side\textsuperscript{4,5}. Comparing pain-induced facial grimacing to facial expressions of emotion is one way to determine which component of the pain experience is most captured by the Mouse Grimace Scale. We hypothesized that grimacing would be lateralized to the left side of the face. Examining lateralization of pain-induced facial grimacing is novel to pain research. We examined the asymmetry of pain-induced facial grimacing in CD-1 mice using inflammatory, neuropathic, and reflexive pain. And we found that pain is expressed predominantly on the right side of the face, contrary to other emotions. Our findings have important implications for the measurement of pain, as characterized by suffering, in non-verbal populations and for application in veterinary care settings.

\textbf{Keywords}: mice, pain, grimacing, lateralization, asymmetry}
        \end{@twocolumnfalse}]
        

\section{Introduction}\label{introduction}

Pain plays an important adaptive function, despite being an aversive experience. Melzack and Casey (1968) identified two components of the pain experience, the sensory-discriminative (SD) and motivational-affective (MA) components\textsuperscript{6}. The SD component reflects the quality, intensity, and location of pain, while the MA component reflects the aversive emotional suffering of pain. The International Association for the Study of Pain recognizes pain as characterized by both a ``sensory and emotional experience''\textsuperscript{7}. Despite recognition of both the SD and MA components, numerous measures of the SD component exist, such as von Frey, which measures mechanical withdrawal thresholds, while measures capturing the MA component remain ill-defined. Evidence suggesting that the Mouse Grimace Scale (MGS) may capture the more elusive MA component of pain stems from lesion studies showing that lesions to limbic regions in the brain associated with emotion processing, such as the amygdala, led to the attenuation of pain-induced facial grimacing\textsuperscript{3}. Considering that suffering and the aversive characteristics of pain are ubiquitous, it is essential to identify well-characterized translational methods that objectively measure these components in pain research.

\subsection{Pain Assessment Scales}\label{pain-assessment-scales}

Self-report questionnaires (e.g., visual analogue scales (VAS), numerical rating scales (NRS), verbal rating scales (VRS)) are often used to capture the subjective pain experience in humans. To reduce self-report bias present in subjective measures, Paul Ekman developed the Facial Action Coding System (FACS): an objective measure capturing changes in facial musculature in response to stimuli\textsuperscript{8,9}. By the late 20\textsuperscript{th} century, pain was finally recognized in infants, and the Neonatal Facial Coding System (NFCS) was adapted from the FACS, enabling pain assessment in infants\textsuperscript{10}.

\subsection{The Mouse Grimace Scale}\label{the-mouse-grimace-scale}

While the FACS and NFCS are limited to use in humans, cross-species generalizability of pain captured through facial expressions emerged when researchers collaborated in the development of the MGS\textsuperscript{11}. Four features of the MGS distinguish it from all other measures in algesiometry-\/-\/-the measurement of pain sensitivity as a response to stimuli (e.g., pressure, heat)\textsuperscript{12}. First, well-established measures of pain capture localized pain (e.g., von Frey); however, most pain syndromes are characterized by \emph{diffuse} pain (e.g., visceral pain)\textsuperscript{12}. The MGS captures pain broadcasted via facial features, i.e., the more commonly reported diffuse type of pain. The second is the \emph{duration} of pain\textsuperscript{12}. Longstanding measures of pain (e.g., hot-plate, tail-flick test) are limited in the sense that they capture pain of short duration (e.g., seconds to minutes). While such measures were pioneering in the field, they do not capture the most clinically relevant type of pain, that is chronic pain lasting months to years. However, the MGS is a measure that captures facial expressions of pain over extended periods of time. The third feature is pain \emph{modality}, more specifically, experimenter-evoked versus spontaneous pain\textsuperscript{12}. A major disadvantage of the established measures of pain is that they capture experimenter-evoked pain rather than spontaneous pain. Spontaneous pain is of greater clinical significance as it relates more closely with chronic pain. The MGS mends this issue by capturing facial expressions of pain in response to spontaneous pain across various modalities (mechanical, thermal, chemical, or electrical) by relying on the broadcast of pain via facial features. The fourth and final component is \emph{outcome measures} (e.g., subjective vs objective)\textsuperscript{12}. Established pain assessment scales capturing the MA component are limited to use in humans due to their requirement for written or verbal communication through subjective questionnaires. The MGS is an objective measure that captures the MA component via the external broadcast of facial features, permitting translation of this model across species. We argue that the MGS is one such tool permitting the preclinical study of the more clinically significant pain-\/-\/-chronic pain characterized as diffuse, spontaneous, and longer lasting.

\subsection{Pain: A Sensation and an Emotion}\label{pain-a-sensation-and-an-emotion}

Pain encompasses both sensory and affective features, however, the more researched of the two remains the sensory component. Recognition of the sensory component is ubiquitous, as demonstrated by pain being recognized as the fifth vital sign in the field of medicine\textsuperscript{13}. However, the affective component of pain, characterized by subjective suffering, remains ill-captured across most species. Most established preclinical models aim to capture affect measure behaviours (e.g., tail suspension and conditional place preference) as proxies for the emotionality of pain\textsuperscript{14}. As research using rodents is characterized by a lack of verbal assessments, we cannot be certain about the emotional component of their pain. Because the MA component of pain is characterized by suffering, arguably the more relevant aspect of the pain experience, it is crucial to clearly define translational measures, such as the MGS, to directly capture this component.

\subsection{Lateralization of Facial Expressions of Emotion}\label{lateralization-of-facial-expressions-of-emotion}

Darwin first noted the asymmetricity of facial expressions of emotion\textsuperscript{15}. Prior studies have shown that facial expressions of emotion (e.g., fear, anger) are predominantly expressed on the left side of the face\textsuperscript{16}. In both human and non-human samples, left asymmetry of emotion was found in facial expressions via third-party interpretation, and in brain areas involved in their expression\textsuperscript{4}. External left-bias for emotional expression is thus also reflected in the internal neural circuitry governing emotion expression. Innateness of left asymmetry in facial expressions of emotion was shown in research examining intensity ratings of spontaneous versus posed expressions of emotion\textsuperscript{17}. Examination of spontaneous versus posed facial expressions of positively and negatively valenced emotions found a stronger left-side display bias for spontaneous versus posed emotions for both happy and sad emotion types. To our knowledge, whether pain is expressed asymmetrically remains unknown. We examined whether spontaneous pain-induced facial grimacing would be lateralized, such that the expression of pain would be stronger on the left or on the right side of the face, using the MGS.

\subsection{Current Study}\label{current-study}

Despite current knowledge on facial grimacing, it remains unclear to what degree the MGS captures the sensory or the emotional aspects of pain. With similar objective grimace scales developed in humans, we have reason to believe that the MGS captures the MA component; therefore, we examined whether pain-induced facial grimacing is lateralized. We hypothesized that grimacing would be lateralized to the left side of the face like facial expressions of emotion. Mice underwent one of five different pain models: zymosan (ZYM), complete Freund's Adjuvant (CFA), acetic acid (AA), carrageenan (CARR), and spared nerve injury (SNI), across three different pain types: neuropathic, inflammatory, and reflexive pain. The SD component of pain was captured using von Frey as an established tool for the assessment of mechanical withdrawal thresholds, with fibres ranging from 0.008 grams to 100 grams of force. Lateralization of MGS scores and von Frey withdrawal thresholds were compared. Using a within-subjects design, mice facial grimacing was coded according to the 5 facial action units (AUs) comprising the MGS coding system.

\section{Materials and Methods}\label{materials-and-methods}

\subsection{Video Capture and Pain Model Induction}\label{video-capture-and-pain-model-induction}

\subsubsection{Overview}\label{overview}

The appropriate research review board, Institutional Animal Care and Use Committee (IACUC), reviewed and approved this research. Adult male and female CD-1 mice on a 12/12-hour light-dark cycle underwent 30-minute baseline video recordings prior to any experimental manipulation. Subsequently, mice were assigned to an experimental group, AA, CFA, ZYM, SNI, or CARR, and injected with the assay. All injections were administered into the hind paw and counterbalanced between left and right, with the exception of AA, which was administered via intraperitoneal injection (IP). IPs of AA were completed by scruffing the mouse and using 1-mL syringes with 26-gauge needle tips to inject into the intraperitoneal cavity. The concentrations and volume of AA for each mouse were determined by mouse weight using a standard dosage formula. CFA, ZYM, and CARR injections were administered into the pad of the hind paw using 1-mL syringes with 26-gauge needle tips. Similarly, the concentration and volume of the pain assay were determined by mouse weight using a standard dosage formula. All injections were completed by a rotating group of trained individuals. SNI surgeries were completed by the same research assistant, ensuring standardization of the procedure. All animals were anesthetized using isoflurane before the surgery. Skin and muscle incisions were made before ligating the terminal tibial and peroneal branches of the sciatic nerve using a non-resorbable silk suture, followed by a resection of a small segment of these two branches\textsuperscript{29}. The sural branch remained intact. In sham controls, no sciatic nerve branches were ligated or resected, they were only exposed. The skin incision was closed in all animals using absorbable silk suture material. Post-injection video recordings were taken following a window of time appropriate for each agent: AA, immediately after injection\textsuperscript{18}; CFA, 2hrs post-injection\textsuperscript{19}; ZYM 30-45 minutes post injection\textsuperscript{20}; SNI, 4 days post injection\textsuperscript{21}; and CARR, 3 hours post-injection\textsuperscript{19}.

\subsubsection{Video Capture}\label{video-capture}

High-definition (HD) video recordings of mice displaying facial grimacing and other relevant pain behaviours were stored on SD cards prior to being uploaded to a server for storage and evaluation. A group of undergraduate students were responsible for taking these videos. Mice were individually placed in single stainless steel and Plexiglas observation cubicles (9 x 5 x 5 cm high) with HD digital video cameras positioned perpendicular to the cubicle and facing the mouse.

\subsubsection{Image Generation and Standard MGS Scoring}\label{image-generation-and-standard-mgs-scoring}

Still images were taken from both baseline and post-model induction videos at 3-minute intervals. 7 images were generated for each 3-minute time interval corresponding to front-facing (unaltered), left and right composite (composite), left and right side (hemiface), and left and right profile view (profile) picture orientations. Still images were cropped so that the body position of the mice was not within the frame. Blinded and randomized scoring of still images was performed to obtain an average baseline and post-induction MGS score for each of the 5 AUs (orbitals, whiskers, ears, cheek and nose bulge), as well as an overall score. A single research assistant was tasked with blindly scoring the stills using the MGS. Analysis of the AA videos was completed separately by counting the number and duration of abdominal constrictions (writhes). Writhing behaviour was defined as repeated constriction of the abdominal muscles, followed by elongation of the body and hind-leg extension\textsuperscript{22}. Five research assistants quantified writhing behaviour by manually counting the number of writhes exhibited by the mice.

\subsubsection{Generation of Composite Images}\label{generation-of-composite-images}

Symmetrical (unaltered) still images of the mice faces were sampled approximately every 3 minutes and were cropped down the y-axis and mirrored to create left-left and right-right facial chimeras. The cropped still images used for chimeras were also used without mirroring, using only the left or right side of the face (hemiface).~

\textbf{\hfill\break
}

\begin{figure}
\centering
\includegraphics[width=\columnwidth]{Figures/image1.png}
\caption{. Lateralization Image Variants: 4 image types of mice faces were collected from video recordings; unaltered front facing images, both left and right side-profile images, both left and right hemiface images, and both left and right composite images made}
\end{figure}

\subsubsection{Scoring Using the Mouse Grimace Scale~}\label{scoring-using-the-mouse-grimace-scale}

The MGS applies a 5-AU scoring system that includes the degree of orbital squinting, nose and cheek bulge, as well as ear and whisker position change\textsuperscript{3}. Rating of AUs is on a scale from 0 to 2 depending on the degree of expression (0 = AU not present, 1 = moderate presence of the AU, 2 = severe presence of the AU). The AUs that were not clearly determined were not scored.

\begin{figure}
\centering
\includegraphics[width=\columnwidth]{Figures/image2.png}
\caption{. Mouse Grimace Scale: The Mouse Grimace Scale assesses the absence, moderate or severe presence of five action units: (1) orbital tightening, (2) nose bulge, (3) cheek bulge, (4) ear position, and (5) whisker change.}
\end{figure}

\subsubsection{Capturing the Sensory Component using von Frey}\label{capturing-the-sensory-component-using-von-frey}

von Frey (vF) was used to capture the sensory component of pain to serve as a methodological control condition against the MGS-\/-\/-believed to capture the affective component of pain. All vF measurements were taken by the same research assistant who remained blind to the conditions. Mice were habituated for 30 minutes before vF measurements were taken.

\section{\texorpdfstring{\textbf{Results}}{Results}}\label{results}

All analyses were conducted using Statistical Package for Social Sciences (SPSS), and graphs were generated using Prism version 7.04. To test our hypothesis that pain-induced facial grimacing is lateralized to the left side of the face, we analyzed raw MGS scores and compared them to the average MGS scores for each AU. We included a total of n = 164 mice in our study, and each data point in the graphs represents one mouse. This analysis was done for each picture orientation: front facing (unaltered), left and right composite (chimera), left and right side only (hemiface), and left and right profile view (profile). Baseline (pre-pain induction) MGS scores were subtracted from post (post-pain induction) MGS scores to obtain a difference/delta score (D MGS). We analyzed these scores using a two-tailed one-sample t-test comparing mean scores to zero. Scores greater than zero indicate a right bias, whereas scores below zero indicate a left bias. The red error bars represent the standard error of the means, whereas the grey areas signify the means. We observed significant lateralization of pain-induced facial grimacing, with pain being expressed predominantly on the right side of the face across all studied angles of the face: composite (t(162)=2.757, \emph{p}=0.0065); hemiface (t(162)=3.685, \emph{p}=0.0003); profile (t(162)=8.486, \emph{p}\textless0.0001); average (t(162)=6.682, \emph{p}\textless0.0001).

\textbf{\hfill\break
}

\begin{figure}
\centering
\includegraphics[width=\columnwidth]{Figures/image3.png}
\caption{. MGS Difference Scores Shown by Face Side: Pain-induced facial grimacing was expressed more strongly to the right side of the face from every facial angle. Composite (t(162)=2.757, p\textless.01, M=.02727, SD=.1262); hemiface (t(162)=3.685, p\textless.001, M=.03461, SD=.1199); profile (t(162)=8.486, p\textless.0001, M=.07984, SD=.1201); average (t(162)=6.682, p\textless.0001, M=.04730, SD=.09037). **p≤0.01, ***p≤0.001, ****p≤0.0001.}
\end{figure}

A two-tailed one-sample t-test was conducted comparing the delta MGS scores for each AU to zero. We observed a significant effect of orbitals (t(161)=2.418, \emph{p}=0.0167), nose bulge (t(161)=3.836, \emph{p}=0.0002), and cheek bulge (t(161)=2.912, \emph{p}=0.0041). However, the ears (t(161)=0.6841, \emph{p}=0.495) and whiskers (t(154)=1.666, \emph{p}=0.0978) AU scores were not significantly different from 0.

\textbf{\hfill\break
}

\begin{figure}
\centering
\includegraphics[width=\columnwidth]{Figures/image4.png}
\caption{. \textbf{MGS Difference Scores by Action Unit:} Mice showed a bias in pain-induced facial grimacing to the right in the eye (t(161)=2418, p\textless.05, M=.006790, SD=.03575), nose (t(161)=3.836, p\textless.001, M=.01056, SD=.1217), and cheek (t(161)=2.912, p\textless.01, M=.01056, SD=.04613). The ears (t(161)=.6841, p=ns, M=.006543, SD=.1217) and whiskers (t(154)=1.666, p\textless.05, M=.01523, SD=.1138), although trending towards right-biased lateralization, did not reach significance, which indicates differential lateralization of grimacing action units. *p≤0.05, **p≤0.01, ***p≤0.001, ns=nonsignificant.}
\end{figure}

A two-tailed one-sample t-test was conducted comparing delta MGS scores depending on the side of injection of the pain assay. There was a significant right-bias in MGS scores for pain-induced facial grimacing, which occurred regardless of the side of pain injection. Pain injection on the right (t(73)=4.370, \emph{p}\textless0.0001), left (t(72)=4.236, \emph{p}\textless0.0001), and non-lateral IP (t(15)=2.831, \emph{p}=0.0127) all caused stronger pain-induced facial grimacing on the right side of the face.

\begin{figure}
\centering
\includegraphics[width=\columnwidth]{Figures/image5.png}
\caption{. MGS Difference Scores Shown by Pain Location: Pain-induced facial grimacing was predominantly expressed on the right side of the face regardless of pain location; right pain (t(73)=4.370, p\textless.0001, M=.04554, SD=.08964), left pain (t(72)=4.236, p\textless.0001, M=.04603, SD=.09284), non-lateral pain (t(15)=2.831, p\textless.05, M=.06125, SD=.08655). *p≤0.05, ****p≤0.0001.}
\end{figure}

A two-tailed one-sample t-test was conducted comparing the delta MGS scores for each pain assay. We observed a significant effect of SNI (t(29)=4.088, \emph{p}=0.0003), AA (t(15)=2.831, \emph{p}=0.0127), ZYM (t(32)=3.075, \emph{p}=0.0043), and CFA (t(59)=3.860, \emph{p}=0.0003). The delta MGS scores for CARR (t(23)=1.407, \emph{p}=0.172) were not significantly different from 0.

\begin{figure}
\centering
\includegraphics[width=\columnwidth]{Figures/image6.png}
\caption{. MGS Difference Scores by Inflammatory Assay: Mice showed a strong lateralization of facial grimacing response to inflammatory assays: SNI (t(29)=4.088, p\textless.001, M=.05333, SD=.07145), AA (t(15)=2.831, p\textless.05, M=.06125, SD=.08655), ZYM (t(32)=3.075, p\textless.01, M=.04182, SD=.07812), and CFA (t(59)=3.860, p\textless.001, M=.05067, SD=.1017), but no lateralization was observed in response to CARR (t(23)=1.407, p=ns, M=.02958, SD=.1030). *p≤0.05, **p≤0.01, ***p≤0.001, ns=nonsignificant.}
\end{figure}

An independent samples t-test, assuming equal variance, was conducted to investigate whether von Frey withdrawal thresholds were lateralized. The side of injury predicted ipsilateral and contralateral withdrawal thresholds congruently, such that a left side injury predicted lower withdrawal thresholds on the ipsilateral (left) side (t(109)=1.509, \emph{p}=0.134) and high withdrawal thresholds on the contralateral (right) side (t(109)=-1.036, \emph{p}=0.302). This result suggests that mechanical hypersensitivity is not lateralized.

\begin{figure}
\centering
\includegraphics[width=\columnwidth]{Figures/image7.png}
\caption{. \textbf{von Frey Withdrawal Thresholds and Site of Injury:} Variance in withdrawal thresholds were congruent with the site of injury. Left side injury would predict lower von Frey withdrawal thresholds on the ipsilateral side and higher von Frey withdrawal thresholds on the contralateral side, suggesting that mechanical hypersensitivity is not lateralized. Ipsi-L vs ipsi-R (t(218)=1.303, p\textless ns, M\textsubscript{ipsi-L}=79.64, M\textsubscript{ipsi-R}=69.69); contra-L vs. contra-R (t(218)-1.160, p\textless ns, M\textsubscript{contra-L}=9.604, M\textsubscript{contra-R}=18.47).}
\end{figure}

\section{Discussion}\label{discussion}

MGS scores for pain-induced facial grimacing were greater for the right side of the face than the left. von Frey withdrawal thresholds were congruent with the side of pain and thus not lateralized. Specifically lower withdrawal thresholds were associated with the pain side whereas higher withdrawal thresholds were associated with the no-pain side. We hypothesized that pain-induced facial grimacing would be lateralized to the left side of the face like other facial expressions of emotion, such as fear. Our results did not support this hypothesis, suggesting that pain does not behave like other facial expressions of emotion. These findings are novel to the field of pain and hold major pre-clinical significance.

Importantly, our results suggest that the MGS captures the more elusive MA component, as suggested by the incongruency between the lateralization of pain-induced facial grimacing and the absence of lateralization of von Frey withdrawal thresholds, which are thought to capture the SD component of pain. This incongruency suggests that the pain measurements MGS and von Frey capture different components of the pain experience, notably the MA and the SD respectively. Characterization of the MGS as the first pre-clinical measure of the MA component of pain opens important avenues in the field, particularly as it pertains to pain assessment in underrepresented, non-verbal populations. We argue that the MGS, due to its ability to capture characteristics resembling that of chronic pain, should serve as a golden standard in pre-clinical and clinical pain research. Utilizing the MGS in pre-clinical pain assessments brings the field one step closer to demystifying the nature of chronic pain.

One possible explanation as to why pain and emotion expression are not congruently lateralized is in their evolutionary underpinnings. Emotions communicate socially meaningful information (e.g., social norms)\textsuperscript{24}, while the physiological experience of pain communicates vital information reflective of survival (e.g., the reflex to remove one's hand from a hot stove), thus reflecting the evolutionary basis for segregated information processing pathways\textsuperscript{25}. Our results reflect this notion as the communication of pain and emotion, indicated by facial grimacing, is differentially lateralized.

\section{Limitations}\label{limitations}

Despite the success of the MGS in detecting facial expressions related to pain, such tools might not fully capture the emotional and psychological dimensions of pain that are present in humans. Emotional suffering in pain is multifaceted, involving not just sensory inputs but cognitive assessments, personal history, and cultural context, all of which contribute to how pain is experienced. Therefore, while the MGS and similar tools can provide valuable data on the sensory aspects of pain (e.g., the physical distress indicated by grimacing), they may not account for the entire spectrum of pain's emotional impact in humans. This study underscores the importance of refining measures of the motivational-affective component in both animal and human models to improve our understanding of pain.

\section{Conclusion}\label{conclusion}

In summary, we found that pain-induced facial grimacing is lateralized to the right side of the face, contrary to expressions of emotion. Segregation of the emotion and pain pathways may reflect evolutionary adaptations, as evidenced by differential lateralization of facial expressions for pain and emotion. This finding highlights the MGS as the only pre-clinical measure translatable across species that effectively captures the MA component of chronic pain. This has critical implications for the treatment of suffering and the adversity of chronic pain.

\section{Acknowledgments}\label{acknowledgments}

I would like to acknowledge Dr. Jeffrey Mogil and PhD candidate Alicia Zumbusch for their estimable guidance throughout the course of this research project. Their invaluable support and assistance served a pivotal role in the execution of this project. I would like to thank my fellow undergraduate students, Gabriel Firanescu, Dana Harrel, Sijie Xu, as well as research assistant 2, Mrs. Susana Sotocinal, for their tremendous help throughout the video-taking and scoring process.

\section{Statement of Contribution}\label{statement-of-contribution}

I had the pleasure of working on a research project, part of a larger whole, under Dr. Jeffrey Mogil and PhD candidate Alicia Zumbusch. Collection of MGS videos along with scoring was assisted by fellow undergraduates and stills were collected on a weekly basis. Alicia Zumbusch helped with the collection of background literature, developed the project methodology and provided guidance towards relevant sources. As the eldest undergraduate, I trained my fellow undergraduates on wet-lab tasks and video still extraction. I conducted the data analysis with the help of Dr. Jeffrey Mogil and Alicia Zumbusch and interpretation of the results was subsequently approved by the latter.

\section{References}\label{references}

\begin{enumerate}
\def\labelenumi{\arabic{enumi}.}
\item
  Melzack, R., \& Wall, P. D. (1965). Pain mechanisms: a new theory.~\emph{Science,}~\emph{150}(3699), 971--979. https://doi.org/10.1126/science.150.3699.971
\item
  Talbot, K., Madden, V. J., Jones, S. L., \& Moseley, G. L. (2019). The sensory and affective components of pain: are they differentially modifiable dimensions or inseparable aspects of a unitary experience? A systematic review.~\emph{Br J Anaesth.},~\emph{123}(2), e263--e272. https://doi.org/10.1016/j.bja.2019.03.033
\item
  Langford, D, J., et al. (2010)\emph{.}~Coding of facial expressions of pain in the laboratory mouse.~\emph{Nat Methods}~7, 447--449. https://doi.org/10.1038/nmeth.1455
\item
  Lindell, A. in \emph{Cerebral lateralization and cognition: Evolutionary and Developmental Investigations of Behavioral Biases} (eds G. S. Forrester, W. D. Hopkins, K. Hudry, \& A. Lindell) 249--270 (Elsevier Academic Press, 2018).~https://doi.org/10.1016/bs.pbr.2018.06.005
\item
  Mandal, M. K., \& Ambady, N. (2004). \emph{Laterality of facial expressions of emotion: Universal and culture-specific influences}.~\emph{Beh Neurol},~\emph{15}(1-2), 23--34. https://doi.org/10.1155/2004/786529
\item
  Melzack, R., \& Casey, K. L. in \emph{The Skin Senses} (eds Kenshalo, D. R.)\emph{\-} 423-443 (Charles C. Thomas, 1968).
\item
  Raja, S. N., et al. (2020). The revised International Association for the Study of Pain definition of pain: concepts, challenges, and compromises. \emph{Pain}, 161(9), 1976-1982. https://doi.org/10.1097/j.pain.0000000000001939
\item
  Prkachin, K. M. (1992). The consistency of facial expressions of pain: a comparison across modalities. \emph{Pain}, 51(3), 297-306. https://doi.org/10.1016/0304-3959(92)90213-U
\item
  Ekman, P., \& Friesen, W. V. (1976). Measuring facial movement. \emph{Environmental Psychology \& Nonverbal Behavior, 1}(1), 56--75. \href{https://psycnet.apa.org/doi/10.1007/BF01115465}{https://doi.org/10.1007/BF01115465}
\item
  Rodkey, E. N., \& Pillai Riddell, R. (2013). The infancy of infant pain research: the experimental origins of infant pain denial.~\emph{J Pain},~\emph{14}(4), 338--350. https://doi.org/10.1016/j.jpain.2012.12.017
\item
  Langford, D. J., Bailey, A. L., Chanda, M. L., Clarke, S. E., Drummond, T. E., Echols, S., Glick, S., Ingrao, J., Klassen-Ross, T., Lacroix-Fralish, M. L., Matsumiya, L., Sorge, R. E., Sotocinal, S. G., Tabaka, J. M., Wong, D., van den Maagdenberg, A. M., Ferrari, M. D., Craig, K. D., \& Mogil, J. S. (2010). Coding of facial expressions of pain in the laboratory mouse.~\emph{Nature methods},~\emph{7}(6), 447--449. doi.org/10.1038/nmeth.1455
\item
  Mogil J. S. (2022). The history of pain measurement in humans and animals.~\emph{Front Pain Res.},~\emph{3}, 1031058. https://doi.org/10.3389/fpain.2022.1031058
\item
  Scher, C., Meador, L., Van Cleave, J. H., \& Reid, M. C. (2018). Moving Beyond Pain as the Fifth Vital Sign and Patient Satisfaction Scores to Improve Pain Care in the 21st Century.~\emph{Pain Manag Nurs.},~\emph{19}(2), 125--129. \url{https://doi.org/10.1016/j.pmn.2017.10.010}
\item
  Farzad Salehpour, Javad Mahmoudi, Saeed Sadigh-Eteghad, Paolo Cassano, Chapter 14 - Photobiomodulation for depression in animal models, Editor(s): Michael R. Hamblin, Ying-Ying Huang, Photobiomodulation in the Brain, Academic Press, 2019, 189-205, doi.org/10.1016/B978-0-12-815305-5.00014-2
\item
  Darwin. C. R. (1872). The expression of the emotions in man and animals. \emph{John Murray, London.}
\item
  Lindell, A. K. (2013). The silent social/emotional signals in left and right cheek poses: A literature review.~\emph{Laterality},~\emph{18}(5), 612-624. https://doi.org/10.1080/1357650X.2012.737330
\item
  Dopson, W. G., Beckwith, B. E., Tucker, D. M., \& Bullard-Bates, P. C. (1984). Asymmetry of facial expression in spontaneous emotion.~\emph{Cortex},~\emph{20}(2), 243--251. https://doi.org/10.1016/s0010-9452(84)80041-6
\item
  Gawade S. P. (2012). Acetic acid induced painful endogenous infliction in writhing test on mice.~\emph{J Pharmacol \& Pharmacother.},~\emph{3}(4), 348. https://doi.org/10.4103/0976-500X.103699
\item
  Ren, K., Dubner, R. (1999). Inflammatory Models of Pain and Hyperalgesia,~\emph{ILAR J.}, \emph{40}(3), 111--118. https://doi.org/10.1093/ilar.40.3.111
\item
  Meller, S. T., \& Gebhart, G. F. (1997). Intraplantar zymosan as a reliable, quantifiable model of thermal and mechanical hyperalgesia in the rat.~\emph{Eur J Pain},~\emph{1}(1), 43--52. https://doi.org/10.1016/s1090-3801(97)90052-5
\item
  Menorca, R. M., Fussell, T. S., \& Elfar, J. C. (2013). Nerve physiology: mechanisms of injury and recovery.~\emph{Hand Clinics},~\emph{29}(3), 317--330. https://doi.org/10.1016/j.hcl.2013.04.002
\item
  Dai, G., et al. (2021). Synergistic interaction between matrine and paracetamol in the acetic acid writhing test in mice.~\emph{Eur J Pharmacol.},~\emph{895}, 173869. https://doi.org/10.1016/j.ejphar.2021.173869
\item
  Matsumiya, L. C., et al. (2012). Using the Mouse Grimace Scale to reevaluate the efficacy of postoperative analgesics in laboratory mice.~\emph{J Am Assoc Lab Anim Sci.},~\emph{51}(1), 42--49.
\item
  van Kleef, G. A., Cheshin, A., Fischer, A. H., \& Schneider, I. K. (2016). Editorial: The Social Nature of Emotions. \emph{Front Psychol., 7}. https://doi.org/10.3389/fpsyg.2016.00896
\item
  Garland E. L. (2012). Pain processing in the human nervous system: a selective review of nociceptive and biobehavioral pathways.~\emph{Prim Care},~\emph{39}(3), 561--571. https://doi.org/10.1016/j.pop.2012.06.013
\item
  Budell, L., Kunz, M., Jackson, P. L., \& Rainville, P. (2015). Mirroring pain in the brain: emotional expression versus motor imitation. \emph{PloS One}, 10(2), e0107526. https://doi.org/10.1371/journal.pone.0107526
\item
  Ji, G., \& Neugebauer, V. (2009). Hemispheric lateralization of pain processing by amygdala neurons.~\emph{J of Neurophysiol.},~\emph{102}(4), 2253--2264. https://doi.org/10.1152/jn.00166.2009
\item
  Roza, C., \& Martinez-Padilla, A. (2021). Asymmetric Lateralization during Pain Processing.~\emph{Symmetry},~\emph{13}(12), 2416. https://doi.org/10.3390/sym13122416
\item
  Shields, S. D., Eckert, W. A., 3rd, \& Basbaum, A. I. (2003). Spared nerve injury model of neuropathic pain in the mouse: a behavioral and anatomic analysis.~\emph{The Journal of Pain},~\emph{4}(8), 465--470. https://doi.org/10.1067/s1526-5900(03)00781-8
\end{enumerate}

\end{document}

\printbibliography
\end{document}