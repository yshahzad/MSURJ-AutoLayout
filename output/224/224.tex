
        \documentclass{msurj}
        \usepackage{multirow}
        \begin{document}
        \twocolumn[
        \begin{@twocolumnfalse}
        \maketitle
        {Lena Hug, Refilwe Mpai}
        {Optimization of Experimental Conditions to Characterize Perineuronal Nets in the Human Cerebellum\strut\\}
        {01/16/2024}
        {Research Article}
        {Department of Psychology, McGill University; Department of Psychiatry, McGill University}
        {Perineuronal nets, child abuse, major depression, cerebellum, RNAscope, immunofluorescence}
        {xyz@mail.mcgill.ca}
        {\label{abstract}

During childhood and adolescence, the brain is highly responsive to external stimuli compared to adulthood. Perineuronal nets (PNNs), play a crucial role during this period by reducing neuroplasticity. These mesh-like structures protect and fortify neural connections between cells. Child abuse includes physical, emotional, and sexual abuse and/or neglect.\textsuperscript{\cite{ref1}} It is consistently associated with negative mental and physical health outcomes, underscoring the importance of identifying risk and resilience factors for effective prevention of such outcomes. Our laboratory focuses on understanding the cellular and molecular neuroanatomy of major depression and the lasting impact of child abuse (CA) on the brain. However, the impact of CA on PNNs remains relatively unexplored. How does CA influence the brain, potentially contributing to negative outcomes in the future? Samples from post-mortem human brain cerebellum were dissected and then used to perform RNAscope experiments to label glutamatergic, GABAergic, and parvalbumin-positive cells, following a brief IF protocol using Wisteria Floribunda Lectin (WFL) to visualize PNNs. The RNAscope protocol was successfully optimized by the addition of normal donkey serum (NDS), manipulation of incubation time, and WFL concentration. PVALB+ mRNA expression was positively identified in Purkinje cells, molecular layer interneurons, and deep cerebellar nucleus (DCN) neurons. SLC17A7+ mRNA expression was evident in granule cells and excitatory projection DCN neurons. GAD1+ mRNA expression was detected in Purkinje cells and inhibitory DCN neurons. These results provide an experimental protocol for future studies investigating the role of PNNs in the human cerebellum. We propose that CA alters the recruitment of PNNs, influencing circuitry and potentially increasing susceptibility to various mental illnesses, including major depressive disorder (MDD). MDD, also called clinical depression, causes a persistent feeling of sadness and loss of interest. \includegraphics[width=\columnwidth]{/Users/yavuz/Desktop/MSURJ AutoLayout/data/ir_tex/224/media/image1.png}This study aims to optimize fluorescent in situ hybridization (FISH, RNA-scope) and immunofluorescence (IF) markers for the localization and phenotyping of PNN-enwrapped neurons in the human cerebellum. This article describes problems we encountered when running experiments and ways to optimize them. As this work is preliminary, it will help develop future protocols for exploring the effects of depression on PNNs and the phenotype of the cells they encircle. Comparing depressed individuals with and without a history of CA with neurologically and psychiatrically healthy controls will allow us to determine whether a history of CA impacts the distribution and density of PNNs.

\ul{Key Words:} Perineuronal nets, Child Abuse, Major Depression, Cerebellum, RNAscope, Immunofluorescence}
        \end{@twocolumnfalse}]
        

\section{Introduction}\label{introduction}

The pivotal periods of childhood and adolescence are characterized by high neuroplasticity, in which the brain is highly adaptable and neuronal connections can undergo significant changes. A child\textquotesingle s environment is thus highly important for their developmental trajectory. Throughout childhood, the brain is especially responsive to external stimuli and is therefore more malleable relative to adulthood. PNNs are key structures that reduce neural plasticity and mark an end to this critical period. These intricated structures of condensed extracellular matrix consist of chondroitin sulphate proteoglycans (CSPGs)\textsuperscript{\cite{ref2}} and lecticans, forming a mesh-like net around specific brain cells. Serving as both protectors and solidifiers, PNNs fortify the connections between cells, ultimately closing the window of critical plasticity.\textsuperscript{\cite{ref3}} In this way, PNNS are also involved in encoding, maintaining, and updating memories.\textsuperscript{\cite{ref4}} Previous studies have shown that the digestion of PNNs in animal models increases plasticity, leading to enhanced memory interference from competing information during the encoding process. But this removal can render neurons vulnerable to damage, particularly in neurodegenerative conditions. \textsuperscript{\cite{ref5}}

CA encompasses physical, emotional, or sexual maltreatment and neglect of a child, leaving enduring impacts on the brain and severe consequences for psychological development. Individuals with both MDD and a history of CA tend to experience more severe diagnoses, recurring episodes, and resistance to treatment.\textsuperscript{\cite{ref6}} Results from the study by Gruhn and Compas (2020) show that CA is associated with poor emotion regulation, increased social avoidance, expression of negative emotion in stressful moments and emotion suppression. Hart and Rubia (2012) use neuroimaging studies to show that childhood maltreatment is associated with neuropsychological impairments in academic achievement, IQ, memory, emotion processing, attention, and response inhibition. Evidence has emerged for maltreatment-associated structural deficits in the adult hippocampus, corpus callosum, anterior cingulate cortex (ACC), orbitofrontal cortex (OFC), and dorsolateral prefrontal cortex (PFC) have all been shown in the literature.\textsuperscript{\cite{ref9}} Moreover, CA is associated with increased recruitment of PNNs in the ventromedial prefrontal cortex in depressed suicides (DS) compared to DS with no history of CA and healthy controls.\textsuperscript{\cite{ref10}} Aside from this study, the impact of CA on PNNs remains relatively unexplored.

Traditionally, the cerebellum (CB) has been primarily associated with maintaining balance and controlling motor functions. Nonetheless, recent studies have begun to uncover additional roles of the CB, such as regulating emotions, suppressing impulsive decision-making, enhancing attention, and facilitating working memory.\textsuperscript{\cite{ref11,ref12}} The CB is increasingly implicated in many psychiatric disorders, including attention deficit hyperactivity disorder, autism spectrum disorders, schizophrenia, bipolar disorder, major depressive disorder, and anxiety disorders.\textsuperscript{\cite{ref13,ref14,ref15}} The vast majority of existing studies looking at psychiatric disorders have employed neuroimaging methods. However, little is known about the cellular and molecular changes in the CB that are caused by psychiatric disorders, as well as the impact of CA on this brain region.

In regions where PNNs have been localized, these structures were shown to surround parvalbumin-expressing (PV) inhibitory neurons. PV neurons have been shown to regulate cortical sensory plasticity during both child development and adulthood.\textsuperscript{\cite{ref10}} Unpublished work from the Mechawar lab has shown that, in healthy individuals with no history of neurological or psychiatric illnesses, PNNs in the deep cerebellar nuclei (DCN) preferentially surround SLC17A7+/PVALB+ cells (\textasciitilde70\%), while a much smaller percentage surround GAD1+/PVALB+ cells (\textasciitilde20\%). The remainder of PNNs are surrounding PVALB- cells that are either GAD1+ or SLC17A7+. SLC17A7 is a gene expressed in glutamatergic cells, GAD1 is a gene expressed in inhibitory cells, and PVALB is expressed in PV+ cells.

PV+ cells play a crucial role in generating and sustaining cortical gamma oscillations, which are essential for the consolidation and retrieval of memories. In the absence of proper regulation of cortical PV+ neuron activity, there is a disruption in the excitatory--inhibitory balance and signal transmission within vital neural networks associated with cognition. Such disruptions that may be implicated in CA could contribute to changes in inhibitory signaling and gamma oscillations implicated in various psychiatric conditions that manifest during childhood and early adulthood.\textsuperscript{\cite{ref16}} As the cerebellum's role in psychiatric conditions is increasingly studied, PV+ neurons in this brain region could influence signal transmission and other cognitive processes.

The study by Tanti et al. (2022) looked at PNN changes in the ventromedial prefrontal cortex (vmPFC). This brain region is involved in emotional learning, which is functionally altered in individuals with a history of child abuse. They investigated how both PNNs and the phenotype of cells they surround are impacted in depressed suicides both with and without a history of childhood, as well as in comparison to psychologically healthy controls, in the CB. We hypothesize that child abuse alters the recruitment of PNNs, affecting the circuitry, and possibly predisposing individuals with such a history to various mental illnesses including MDD. With the results we have obtained in the deep cerebellar nuclei, we anticipate that the PNNs will be reduced in DS-CA compared to the other groups, resulting in asynchronous cell firing in the DCN and impacting communication between the CB and other regions involved in MDD like the vmPFC, hippocampus, and amygdala. We anticipated DS (without CA) condition to look like the control condition, a hypothesis aligned with Tanti et al (2022).

As this work is still in progress, this study aims to establish a working RNAscope and IF protocol to investigate how both cerebellar PNNs and the phenotype of cells they surround are impacted in depressed suicides with or without a history of childhood maltreatment compared to neurologically and psychiatrically healthy controls. By optimizing fluorescent in situ hybridization (FISH, RNA-scope) and immunofluorescence (IF) markers for the localization and phenotyping of PNN-enwrapped neurons in the human CB, we hope to provide a strong protocol for future studies.

\section{Materials and Methods}\label{materials-and-methods}

To test our hypothesis, we dissected human post-mortem brain samples and performed fluorescent \emph{in situ} hybridization (FISH; RNAscope) experiments to label vesicular glutamate transporter 1 (SLC17A7+), glutamate decarboxylase 1 (GAD1+), and Parvalbumin (PVALB+) cells. We then performed a short immunofluorescence (IF) protocol using Wisteria Floribunda Lectin (WFL) to visualize PNNs and looked at our samples using a VS120 Olympus Slide scanner microscope. Nuclei 6-diamidino-2-phenylindole (DAPI) helped visualize nuclear DNA in cells, as it is a marker for membrane viability. In theory, WLF (green) should appear with PVALB (white dots), as it surrounds neuronal cells (with a blue nucleus). The association of WFL and PVALB in the result should indicate PNN's presence.\textsuperscript{\cite{ref17}} See Figure 1B of the cortex.

\subsection{Human post-mortem brain samples}\label{human-post-mortem-brain-samples}

The lab works in close relation with the Douglas-Bell Canada Brain Bank in Montréal. Brain samples underwent a series of procedures prior to inclusion in this project. Consent from the next of kin and the Quebec Coroner's Office was secured and helped us conduct standardized psychological autopsies for gathering phenotypic information. Individuals with neurological or neurodegenerative disorders were excluded. After examination of medical charts and Coroner records, case and control groups were defined. Additionally, proxy-based interviews with individuals closely acquainted with the deceased obtained from hospitals, the Coroner's office, and social services provided supplementary information. To qualify individuals according to the DSM-IV diagnostic criterion, a panel of clinicians listed and revised the criterion based on questionnaires covering sociodemographic characteristics, social developmental history, DSM-IV axis I diagnostic information, and behavioural traits. This was conducted before the death of patients or by questioning their relatives. Using Childhood Experience of Care and Abuse (CECA)\textsuperscript{\cite{ref18}} interviews, clinicians were able to identify the presence of severe child abuse, focusing on sexual and physical abuse and neglect. Toxicological assessments and medication histories were also obtained. Subject characteristics are described in Table 1. The REB of the protocol is \#~IUSMD-20-35. All individuals were Caucasian. The method of death wasn't shared with the researchers. Therefore, we only have information on whether the individual died by accident, natural causes, or suicide.

\begin{table}[htbp]
\centering
\small
\caption{Table 1*CTRL: Control; DS: Depressed suicide; DS-CA: Depressed suicide with a history of child abuse.}
\begin{tabularx}{\columnwidth}{@{}
  >{\raggedright\arraybackslash}p{(\linewidth - 6\tabcolsep) * \real{0.1893}}
  >{\raggedright\arraybackslash}p{(\linewidth - 6\tabcolsep) * \real{0.1173}}
  >{\raggedright\arraybackslash}p{(\linewidth - 6\tabcolsep) * \real{0.3450}}
  >{\raggedright\arraybackslash}p{(\linewidth - 6\tabcolsep) * \real{0.3484}}@{}}
\toprule\noalign{}
\begin{minipage}[b]{\linewidth}\raggedright
Table 1
\end{minipage} & \begin{minipage}[b]{\linewidth}\raggedright
CTRL
\end{minipage} & \begin{minipage}[b]{\linewidth}\raggedright
DS
\end{minipage} & \begin{minipage}[b]{\linewidth}\raggedright
DS-CA
\end{minipage} \\
\midrule\noalign{}

Axis 1 Diagnostic & Nill & MDD / DD-NOS (2) & MDD -- DD-NOS(1) \\
Age (years) & 45 \(\pm\)5.27 & 49\(\pm\)3.29 & 41\(\pm\)3.17 \\
Sex (F/M) & 2F/13M & 2F/13M & 4F/11M \\
Post-mortem interval (hours) & 37 \(\pm\)\emph{4.66} & 46 \(\pm\)\emph{8.16} & 46 \(\pm\)\emph{6.14} \\
Substance dependence & 1 & 5 & 6 \\
Tissue pH & 6 \(\pm\)\emph{0.1} & 7 \(\pm\)\emph{5.3} & 7 \(\pm\)\emph{0.1} \\
Medication & SSRI (1) & Benzodiazepines (5), SSRI (3), SNRI (1), Antipsychotics (2), TCA (1), Antimanic (1) & Benzodiazepines (3), SSRI (4), SNRI (1), Antipsychotics (3), Antimanic (1), Opiate (1) \\
\end{tabularx}
\end{table}

While depression is almost 2 times more likely to occur in females than in males, completed suicides are more common in men. Therefore, our depressed suicides tend to include more men than women. In addition, we are using mostly the same cohort used in Tanti et al. (2022). Keeping the same subjects helps us to compare the changes in PNNs in both the vmPFC\textsuperscript{\cite{ref10}} and the CB. This way, we can better understand the impact of child abuse on PNNs in different regions at the same time.

\subsection{Tissue dissections}\label{tissue-dissections}

The dissections were performed by expert brain bank staff on fresh-frozen 1 cm-thick sections, using a human brain atlas. Sections of the deep cerebellar nuclei (DCN) were dissected at the level of the dentate nucleus. A section of the cerebellar cortex was used for comparison. A ventromedial prefrontal cortex (vmPFC) region was used as a positive control but is not displayed in the figures. Samples were kept frozen at -80°C until cryosection.

\subsection{RNAscope and Immunofluorescence}\label{rnascope-and-immunofluorescence}

Since we sought to uncover the phenotype of cells surrounded by PNNs, immunofluorescence was limiting for various reasons. Due to the loss of PV antigenicity that occurs during the freezing process, antibodies against PV neurons don't work in frozen tissue. On the other hand, the markers for PNNs don't work in long-term fixed tissue. Due to the time constraint of this project, using fresh-fixed tissue was not feasible as it is dependent on how often a brain is received (average: 1 -- 3 a week). Moreover, inhibitory interneuron markers commonly used in the cerebrum do not work as well in the CB. To circumvent these issues, RNAscope was a logical step as we could label PNNs and all our cells of interest in the tissue we had available.

Frozen unfixed blocks were cut serially into 10 μm sections using a cryostat and collected on SuperFrost charged slides. The slides were stored at -80°C until they were used for experimentation. Advanced Cell Diagnostics RNAscope probes and reagents were used to perform the RNAscope experiment, following the manufacturer's instructions. Upon removal from the freezer, the sections were initially fixed in cold (4°C) 10\% neutral buffered formalin for 15 minutes. Subsequently, they underwent dehydration through increasing gradients of ethanol baths (75\%, 95\%, and 100\%) and were air-dried for 5 minutes. To quench endogenous peroxidase activity, a 10-minute incubation at room temperature was carried out using 3\% hydrogen peroxide in PBS. Following this, the designated probes: Hs-GAD1 (no. 573061.o1.C3); Hs-SLC17A7 (no.415611); Hs-PVALB (no.422181.C2) were hybridized for 2 hours at 40°C in a humidity-controlled oven. Due to limitations of the number of channels that can be included in one experiment, we had two conditions. The first was PVALB+SLC17A7+, where PVALB was diluted in the C1 SLC17A7. The second was PVALB+GAD1+, where both probes were diluted in probe diluent. During our troubleshooting, we included a third condition which served as a negative control, with only probe diluent and opal dyes. If our conditions with the active probes looked like the negative control, we concluded that further optimization of the experiment was necessary. Amplifiers were added using the proprietary AMP reagents, and the signal was visualized through probe-specific HRP-based detection by tyramine signal amplification with Opal dyes (Opal 570 and Opal 690 for SLC17A7/GAD1 and PVALB respectively; Perkin Elmer) diluted 1:500.

We then conducted a short immunofluorescence measure using biotinylated Wisteria Floribunda Lectin (WFL). Slides were rinsed in PBS and incubated in a solution of WFL (1:500) diluted in PBS/0.2\% Triton-X/5\% normal donkey serum (NDS) for 15 minutes at room temperature. Slides were rinsed again with PBS and incubated in a secondary solution of Alexa 488 conjugated streptavidin (1:500) diluted in PBS/0.2\% Triton-X/5\% NDS to detect PNNs. During our optimization process, we made changes to the length of incubations, the percentage of NDS, as well as the concentration of WFL used. Sections were rinsed. Endogenous autofluorescence from lipofuscin and cellular debris was quenched with Trueblack (Biotium), omitted for tissues used for intensity measurements. Sections were mounted on SuperFrost charged slides and cover slipped with Vectashield mounting medium (Vector Laboratories, H-1800).

\subsection{Microscope and Image analysis}\label{microscope-and-image-analysis}

Image acquisitions were performed on a FV1200 laser scanning confocal microscope using a 10X objective. Charged slides with two sequential sections per subject were used, where we visualized DAPI (405), WFL with Streptavidin-Alexa 488, either GAD1 or SLC17A7 with Opal dye 570, and PVALB with Opal dye 690. The images served for cell counting and overlap.

\section{Results}\label{results}

The current investigation improves the existing experimental procedure for examining the influence of suicidal depression and childhood maltreatment on PNNs and neurons in the deep cerebellar nuclei. This methodology is a refinement of two previous studies carried out in the lab. We successfully optimized an RNAscope protocol. Our initial trial involved negative control slides, visually distinct from slides where the RNAscope probes were applied. This distinction provides us with confidence in the authenticity of the observed staining.

Specifically, PVALB+ mRNA expression was identified in Purkinje cells, molecular layer interneurons, and DCN neurons. SLC17A7+ mRNA expression was evident in granule cells and excitatory DCN neurons. GAD1+ mRNA expression was detected in Purkinje cells and inhibitory DCN neurons (see Figure 1). The absence of staining resembling the structure of PNNs suggests that the morphology of PNNs is not present. We can tell that the WFL staining was suboptimal in Figure 1, as the staining is similar to non-specific lipofuscin staining.

\begin{figure}
\centering
\includegraphics[width=\columnwidth]{Figures/image2.png}
\caption{Figure 1 Representative image of 2 males depressed suicides with no history of CA}
\end{figure}

(A) DCN merge: DAPI marker of cell nuclei in blue, SLC17A7 marker of excitatory cell + in pink, PVALB marker of PV neurons in white, WFL marker of PNN in green.

(B) Representative image of the cerebellar cortex as a comparison for staining validity. Coronal section corresponding to Brodmann areas 11 and 12. We can see PNN are present by the superposition of PVALB and WFL signals.

50 µm scale bars

The validity of the staining was verified using the cerebellar cortex as a comparison, as more is known about the expected staining pattern of these probes than in the DCN. We were able to optimize the WFL IF separately but encountered problems when the IF was conducted in combination with RNAscope experiments. During the initial RNAscope attempt, WFL was diluted at a ratio of 1:500 in PBS + 0.2\% Triton-X (blocking solution) for a 15-minute incubation, followed by the dilution of conjugated streptavidin 488 in the same solution, without the addition of TrueBlack. In this condition, WFL staining resembled the staining pattern seen in the negative control. The expected PNN-patterned staining was not seen. Moreover, in the absence of TrueBlack, there was a lot of background lipofuscin, especially in the 488 channel. Subsequent IF experiments excluding the RNAscope steps but including TrueBlack, resulted in tissue dissolution, necessitating repetition. In the revised experiment, 10\% neutral buffered formalin (NBF) and ethanol dehydration steps at increasing concentrations were introduced to enhance tissue adhesion to the slides. As these are the first steps of an RNAscope experiment there was no concern that this would negatively affect staining or tissue quality.

To further optimize the experiment, two other variables were manipulated: the concentration of WFL (1:250 vs. 1:500, settling on 1:500) and the incubation duration (15 mins vs. 30 mins). Previous IF experiments conducted over two days in the lab using WFL used a concentration of 1:1000. As the goal was to conduct the IF in a short amount of time following the RNAscope, we increased the concentration to compensate for the reduced time. 1:1000 was not sufficient to examine PNNs with the reduced incubation time. PNNs could be visualized well at 1:500, therefore that was the concentration we chose to proceed with. There was no meaningful difference in staining quality between a 15-minute and a 30-minute incubation at room temperature. We therefore chose to proceed with the shortest possible time, which was 15 minutes. When we included TrueBlalck following the formalin and alcohol steps, the endogenous lipofuscin was quenched, allowing us to better differentiate true signal from noise.

After this, 5\% normal donkey serum (NDS) was incorporated into the blocking solution to mitigate non-specific antibody binding, along with the inclusion of NBF and ethanol steps. This, in combination with TrueBlack, allowed for less background non-specific staining. Additionally, a positive control region, the ventromedial prefrontal cortex (vmPFC), known for its high PNN density in layers 3-5 was introduced. This served to distinguish whether the issue lay with the lectin (indicated by an absence of WFL+ PNNs in the vmPFC) or was specific to the CB (WFL+ PNNs in the vmPFC but not in the CB). Notably, the staining was successful (see Figure 2). Upon repetition, and incorporating the RNAscope, the experiment failed without clear reasons. Consequently, further optimization of the IF in conjunction with the RNAscope will be pursued.

\begin{figure}
\centering
\includegraphics[width=\columnwidth]{Figures/image3.png}
\caption{Figure 2 \textbf{Representative figure of WFL in DCN after IF optimization}\\
WFL worked with the addition of 5\% NDS and TrueBlack\\
50 µm scale bars}
\end{figure}

\section{Discussion}\label{discussion}

The present study is ongoing, and the results outlined are preliminary and will be used to further optimize the methodology. The concentration of WFL and incubation length have been manipulated in this study. However, future steps include examining variables such as the concentration of Triton-X (or perhaps omitting it) or the incubation temperature, to help speed up the reaction. WFL (also known as WFA) has been used, in the past, as a marker for PNNs in the CB. Unpublished work from the Mechawar lab has used both WFL and ACAN and shown significant results. In unpublished work conducted in controls, we labelled PNNs using both WFL and ACAN to examine whether there were differences in staining patterns. As both markers were good quality, and there was almost a complete overlap of the two markers, we then proceeded with WFL, the more canonical marker.

The optimization of our experimentations should provide insightful results. This molecular biology work will serve as a breeding ground for future research on the subject. Once optimization is complete, the PNN population will be evaluated under different conditions, and their role will be better assessed. Following optimization, the complete cohort will be processed in four batches, and all slides will be subjected to scanning using a slide scanner. PNNs, glutamatergic cells (SLC17A7), GABAergic cells (GAD1), and PV cells will undergo quantification through QuPath. The densities and extent of overlap between each marker will then be compared across three groups: depressed suicides (DS), depressed suicides with a history of child abuse (DS-CA), and controls (CTRL). The outcomes of this analysis will enable us to address our hypothesis. Future research could use Qupath, following the acquisition of images using a slide scanner to visualize the entire section of tissue.

We hypothesize a reduction in PNNs among those with a history of suicidal depression and child abuse (DS-CA) compared to other groups. This reduction may lead to asynchronous cell firing in the DCN, affecting communication between the CBand other regions implicated in MDD, such as the ventromedial prefrontal cortex (vmPFC), hippocampus, and amygdala.

What if we could manipulate PNN\textquotesingle s expression to change those maladaptive responses in later adulthood? Through meticulous adjustment, there exists the potential to disrupt maladaptive memories, as evident in conditions like post-traumatic stress disorder or drug addiction.\textsuperscript{\cite{ref20}} In the context of age-related cognitive decline and dementia, PNNs may play a pivotal role in fortifying memories or even restoring impaired memory with careful modulation.\textsuperscript{\cite{ref21}} Several studies have shown the significance of epigenetics, pointing to disruptions in the HPA axis, plasticity, and myelination as changes observed in individuals with a history of CA.\textsuperscript{\cite{ref22}} These converging molecular components can be affected by CA and increase the susceptibility to mental disorders in adulthood, and should thus be investigated further. As we know that psychotherapy helps to counteract the impact of CA, different types of interventions (e.g., biological/pharmaceutical) for individuals who experienced child abuse are important additions to the research canon.\textsuperscript{\cite{ref23}} This study is one of the first to look at cellular and molecular substrates of depression and child abuse, especially in the CB. The impact of CA on PNNs in postmortem tissue is a novel field that has much to reveal.

\section{Acknowledgements}\label{acknowledgements}

I thank Dr. Naguib Mechawar for this opportunity in his laboratory. Thank you again to MA Davoli and the rest of the team for the inclusivity and the positivity. The work was supported by the Douglas Hospital of Montreal, the McGill University and the Brain Bank of Canada. \emph{This work was supported by a CIHR Project Grant to NM and by HBHL and Brain Canada Platform support grants to the Douglas-Bell Canada Brain Bank.}

\section{}\label{section}

\printbibliography
\end{document}